\documentclass{article}
\usepackage{ctex,soul,float,listings,enumerate,hyperref,url,amsfonts,amsmath,graphicx,multirow}
\usepackage{xcolor,tocloft,theorem,numerica,amsmath,mathrsfs}
\usepackage{changes}
\usepackage{fancyhdr}
%%%%%%%%%%%%%%%%%%%%%%%%
\definecolor{AnatationColor}{RGB}{0,139,0}
\lstset{
    backgroundcolor = \color{white},    % 背景色:白色
    basicstyle = \small\ttfamily,           % 基本样式 + 小号字体
    rulesepcolor= \color{white},             % 代码块边框颜色,白色
    breaklines = true,                  % 代码过长则换行
    numbers = left,                     % 行号在左侧显示
    numberstyle = \small,               % 行号字体
    keywordstyle = \color{blue},            % 关键字颜色
    commentstyle =\color{AnatationColor},        % 注释颜色
    stringstyle = \color{red!100},          % 字符串颜色
    frame = shadowbox,                  % 用(带影子效果)方框框住代码块
    showspaces = false,                 % 不显示空格
    columns = fixed,                    % 字间距固定
    %escapeinside={<@}{@>}              % 特殊自定分隔符:<@可以自己加颜色@>
    morekeywords = {as},                % 自加新的关键字(必须前后都是空格)
    deletendkeywords = {compile}        % 删除内定关键字;删除错误标记的关键字用deletekeywords删!
}
\hypersetup{
    colorlinks=true,
    linkcolor=red,
    filecolor=blue,      
    urlcolor=blue,
    citecolor=cyan,
}
\newtheorem{definition}{定义}
\graphicspath{{./Image/}}
%%%%%%%%%%%%%%%%%%%%%%%%
\author{ZeitHaum}
\date{\today}
\title{数学-恒等式}
\pagestyle{fancy}
%%%%%%%%%%%%%%%%%%%%%%%%
\begin{document}
    \pagenumbering{Roman}
    \maketitle
    \newpage 
    \tableofcontents
    \newpage
    \setcounter{page}{1}
    \pagenumbering{arabic}
    \section{$\displaystyle{\sum_{i=1}^{n}\frac{1}{i}} = \sum_{i=1}^{n}(-1)^{i-1}\frac{1}{i}\binom{n}{i}$}
    差分法证明,记右式为$f(n)$,左式为$g(n)$。
    \begin{align*}
        f(n+1) - f(n) &= \displaystyle \sum_{i = 1}^{n+1} (-1)^{i-1} \frac{1}{i}\binom{n+1}{i} - \sum_{i=1}^{n}(-1)^{i-1}\frac{1}{i}\binom{n}{i} \\
        &= \displaystyle \sum_{i=1}^{n+1}(-1)^{i-1} \frac{1}{i}(\binom{n}{i} + \binom{n}{i-1}) - \sum_{i=1}^{n}(-1)^{i-1}\frac{1}{i}\binom{n}{i} \\
        &= \displaystyle \sum_{i=1}^{n+1}(-1)^{i-1} \frac{1}{i} \binom{n}{i-1} + (-1)^{n+1} \frac{1}{n+1} \binom{n}{n+1}\\
        &= \displaystyle \sum_{i=0}^{n}(-1)^i \frac{1}{i+1} \binom{n}{i} + 0 \\
        &= \displaystyle \sum_{i=0}^{n+1}(-1)^i \frac{1}{i+1} \binom{n}{i} \\ 
    \end{align*}
    通过差分将上式化为比较简单的式子,接下来处理每一项中的$\frac{1}{i+1}$部分,注意到
    \begin{align*}
        \frac{1}{i+1} \binom{n}{i} &= \frac{1}{i+1} \frac{n!}{(n-i)!i!} \\
        &= \frac{n!}{(n-i)!(i+1)!} \\
        &= \frac{(n+1)!}{(n-i)!(i+1)!} \frac{1}{n+1} \\
        &= \binom{n+1}{i+1} \frac{1}{n+1}
    \end{align*}
    代回,得
    \begin{align*}
        f(n+1) - f(n) &= \displaystyle \sum_{i=0}^{n+1}(-1)^i \frac{1}{i+1} \binom{n}{i} \\ 
        &= \displaystyle \sum_{i=0}^{n+1}(-1)^i \frac{1}{n+1} \binom{n+1}{i+1} \\
        &= \displaystyle \frac{1}{n+1} (\sum_{i=1}^{n+2} (-1)^{i-1} \binom{n+1}{i} -1 + 1)\\
        &= \displaystyle \frac{1}{n+1} (\sum_{i=1}^{n+1} (-1)^{i-1} \binom{n+1}{i} + (-1)^{-1}\binom{n+1}{0} +1)\\
        &= \displaystyle \frac{1}{n+1} (\sum_{i=0}^{n+1} (-1)^{i-1} \binom{n+1}{i} + 1)\\
        &= \displaystyle \frac{1}{n+1} (-\sum_{i=0}^{n+1}\binom{n+1}{i} (-1)^i 1^{n+1-i} + 1) \\
        &= \displaystyle \frac{1}{n+1} (-(-1+1)^{n+1} + 1) \\
        &= \displaystyle \frac{1}{n+1}. 
    \end{align*}
    
    因此有$f(n+1) - f(n) = g(n+1) - g(n)$.
    
    考虑$n = 1$时,$f(1) = g(1) = 1$.
    
    所以$f(n) = g(n)$。

    QED.

\end{document}
