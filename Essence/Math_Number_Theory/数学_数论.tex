\documentclass{article}
\usepackage{ctex,soul,float,listings,enumerate,hyperref,url,amsfonts,amsmath,graphicx,multirow}
\usepackage{xcolor,tocloft,theorem,numerica,amsmath,mathrsfs}
\usepackage{changes}
\usepackage{fancyhdr}
%%%%%%%%%%%%%%%%%%%%%%%%
\definecolor{AnatationColor}{RGB}{0,139,0}
\lstset{
    backgroundcolor = \color{white},    % 背景色:白色
    basicstyle = \small\ttfamily,           % 基本样式 + 小号字体
    rulesepcolor= \color{white},             % 代码块边框颜色,白色
    breaklines = true,                  % 代码过长则换行
    numbers = left,                     % 行号在左侧显示
    numberstyle = \small,               % 行号字体
    keywordstyle = \color{blue},            % 关键字颜色
    commentstyle =\color{AnatationColor},        % 注释颜色
    stringstyle = \color{red!100},          % 字符串颜色
    frame = shadowbox,                  % 用(带影子效果)方框框住代码块
    showspaces = false,                 % 不显示空格
    columns = fixed,                    % 字间距固定
    %escapeinside={<@}{@>}              % 特殊自定分隔符:<@可以自己加颜色@>
    morekeywords = {as},                % 自加新的关键字(必须前后都是空格)
    deletendkeywords = {compile}        % 删除内定关键字;删除错误标记的关键字用deletekeywords删!
}
\hypersetup{
    colorlinks=true,
    linkcolor=red,
    filecolor=blue,      
    urlcolor=blue,
    citecolor=cyan,
}
\newtheorem{definition}{定义}
\graphicspath{{./Image/}}
%%%%%%%%%%%%%%%%%%%%%%%%
\newtheorem{lemma}{引理}
\newcommand{\dis}{\text{dis}}
\newcommand{\ans}{\text{ans}}
\newcommand{\lca}{\text{lca}}
\newcommand{\fa}{\text{fa}}
\newcommand{\size}{\text{size}}
\newcommand{\Inf}{\text{inf}}
\newcommand{\Else}{\text{else}}
\newcommand{\inputcppfile}[1]{\lstinputlisting[language=c++]{#1}}
\renewcommand{\abs}{\text{abs}}
\renewcommand{\top}{\text{top}}
\renewcommand{\gcd}{\text{gcd}}
\renewcommand{\mod}{\text{ mod }}
%%%%%%%%%%%%%%%%%%%%%%%%
\title{数学-数论}
\author{ZeitHaum}
\date{\today}
\pagestyle{fancy}
%%%%%%%%%%%%%%%%%%%%%%%%
\begin{document}
    \pagenumbering{Roman}
    \maketitle
    \newpage 
    \tableofcontents
    \newpage
    \setcounter{page}{1}
    \pagenumbering{arabic}
    \section{互质勾股数的生成方法}
    问题描述: \emph{构造以下集合:}$S = \{(x^2-y^2,2xy,x^2+y^2)| x>y \text{且}x,y\in \mathbb{N}^{+}\}, S' = \{(a,b,c)| a^2+b^2=c^2 \text {且} a,b,c \in \mathbb{N}^{+}\}$,
    \emph{求证以下两个命题:}
    
    \begin{enumerate}[子命题1.]
        \item 在不考虑元组内元素顺序的情况下, 对于$S$中的任意元素$S_i$,都有$S_i \in S'$.
        \item 在不考虑元组内元素顺序的情况下, 对于$S'$中的任意元素$S_i'$,都存在$k\in \mathbb{N}^{+}$使得 $kS_i'\in S$注: 对于3元组$(a,b,c)$, 定义$k(a,b,c) = (ka,kb,kc)$。 
    \end{enumerate}
    
    \subsection{证明}   

    对于命题1,有
    \begin{align*}
        (x^2 - y^2)^2 + (2xy)^2 &=  (x^4 + y^4 - 2x^2y^2) + 4x^2y^2\\
        &= x^4 + y^4 + 2x^2y^2 \\
        &= (x^2 + y^2)^2 
    \end{align*}

    且显然$x^2-y^2,2xy,x^2+y^2 \in \mathbb{N}^{+}$,于是子命题1得证。

    对于命题2, 令$x = b, y = c-a, k = 2y = 2(c-a)$, 

    根据三角形性质,显然有$b > c - a, k \mathbb{N}^{+}$,
    
    所以$(x^2-y^2,2xy,x^2+y^2) \in S$.

    又
    \begin{align*}
        x ^ 2 - y^2 &= b^2 - (c - a)^2 \\
        &= (c^2 - a^2)- c^2 - a^2 + 2ac \\
        &=  2ac - 2a^2 \\
        &= ka,
    \end{align*}

    \begin{align*}
        2xy = kb,
    \end{align*}

    \begin{align*}
        x^2 + y^2 &= b^2 + (c - a)^2\\
        &= c^2 - a^2 + c^2 + a^2 -2ac \\
        &= 2c^2 - 2ac  \\
        &= kc.
    \end{align*}

    所以$(x^2-y^2,2xy,x^2+y^2) = (ka,kb,kc) \in S'$, 证毕。

    \subsection{启示}
    以上两个定理说明通过$S$的构造方法可以显示的求出所有互质的勾股数,只需枚举$x,y$求得$(x^2-y^2,2xy,x^2+y^2)$再让每个数除以三个数的最大公因数即可。


    \section{奇数平方和}

    \emph{求证: 两个奇数的平方和不可能为完全平方数。}

    \emph{证明:} 

    可以通过反证法证明:

    假设$a,b$都为奇数,所以$c^2 $为偶数。
    
    不妨设$a = 2p+1, b = 2q + 1, (p,q\in \mathbb{N})$, 所以
    
    $$a^2 = (2p+1)^2 = 4p^2 + 4p + 1$$

    所以$a^2 \mod 4 = 1, b^2 \mod 4 = 1$.

    所以$a^2 + b^2 \mod 4 = 2$.

    又$a^2 + b^2 = c^2$是完全平方数,所以$a^2 + b^2 \mod 4 = 0$.

    $a^2 + b^2 \mod 4 = 2$和$a^2 + b^2 \mod 4 = 0$矛盾,因此得证。

\end{document}
